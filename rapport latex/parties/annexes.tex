\chapter{Extraits du modèle de données}
\label{chap:annexeA}
\hspace{0pt}

\lstinputlisting[language=SQL, firstline=1, lastline=10, title={......}]{ressources/textuelles/request_pokemon.sql}
\lstinputlisting[language=SQL, firstline=725, lastline=727, caption={Extrait des requêtes de création de la table "pokemon"}]{ressources/textuelles/request_pokemon.sql}

\lstinputlisting[language=SQL, firstline=1, lastline=10, title={......}]{ressources/textuelles/request_team.sql}
\lstinputlisting[language=SQL, firstline=106, lastline=108, caption={Extrait des requêtes de création de la table "team"}]{ressources/textuelles/request_team.sql}

\lstinputlisting[language=XML, firstline=1, lastline=20, title={......}]{ressources/textuelles/moves.xml}
\lstinputlisting[language=XML, firstline=5604, lastline=5613, caption={Extrait du fichier "moves.xml"}]{ressources/textuelles/moves.xml}

\lstinputlisting[language=XML, firstline=1, lastline=10, title={......}]{ressources/textuelles/pokemons.xml}
\lstinputlisting[language=XML, firstline=2500, lastline=2504, caption={Extrait du fichier "pokemons.xml"}]{ressources/textuelles/pokemons.xml}

\lstinputlisting[language=XML, firstline=1, lastline=15, title={......}]{ressources/textuelles/teams.xml}
\lstinputlisting[language=XML, firstline=1912, lastline=1918, caption={Extrait du fichier "teams.xml"}]{ressources/textuelles/teams.xml}

\chapter{Pokémon, kezako ?}
\label{chap:annexeB}
\hspace{0pt}

Les pokémons, abréviation de pocket monster, sont des créatures peuplant un monde imaginaire décrit par les jeux vidéo et animes/mangas du même nom. La franchise, créée en 1995, fait toujours figure de pilier du jeu vidéo et est l'un des emblèmes de Nintendo.

Dans ce monde imaginaire, les pokémons s'apparentent à des animaux réels ou à un mélange de plusieurs animaux réels. Ces créatures sont dotées de capacités (en anglais : moves) qui leur permettent de se livrer à des combats. Elles peuvent être sauvages, et s’attaquer aux personnes qu’elles rencontrent, ou dressées et sous la tutelle d’un dresseur. Ces dresseurs sont des personnes dont l’occupation principale consiste à découvrir et capturer de nouveaux pokémons, et à se livrer des combats par l’intermédiaire de leur équipe de pokémons.

Durant un combat, chaque dresseur ne peut avoir plus de 6 pokémons. Ils se déroulent sous la forme de tours, durant lesquels généralement 1 pokémon de chaque équipe participe au combat, à l’aide de ses capacités. Le dresseur dont tous les pokémons sont mis K.O. perd le combat.

Il existe plusieurs espèces de pokémon. L’espèce est identifiée par un numéro. Elle détermine l’apparence du pokémon, son nom par défaut (ex : Pikachu, Dracaufeu, etc.), son ou ses types (ex : feu, ténèbres, plante, combat, psy…), les capacités qu’il pourra apprendre (ex : charge, tonnerre, ronflement…), et toute caractéristique de base qui le définit.

Le nombre de capacités qu’un pokémon connnaît à un instant donné est limité à 4. Passée cette limite, il doit supprimer une capacité pour en apprendre une nouvelle. Les capacités d’un pokémon ont également un type, qui peut tout à fait différer du type ou des types du pokémon. Par exemple un Dracaufeu, possédant les types feu et vol, peut tout à fait apprendre l’attaque “casse-brique” de type combat.

Un pokémon peut gagner de l’expérience en mettant K.O. un autre pokémon. Ceci lui permettra d’augmenter de niveau et ainsi ses statistiques pour devenir plus efficace. Le gain d’expérience est défini par une base, définie par son espèce.

Les concepts d’expérience, de capacités à oublier pour en apprendre de nouvelles et de “tours” lors des combats sont propres aux jeux plus qu’aux mangas et animes Pokémon.

Enfin, chaque dresseur est équipé d’une encyclopédie appelée “pokédex” lui permettant de collecter des informations sur les pokémon qu’il rencontre. Ce pokédex constitue le bestiaire du monde de pokémon. Chaque pokémon rencontré à un moment donné par le dresseur y sera décrit (type(s), nom, attaques…) et trié dans l’ordre de leur numéro.

\chapter{Schéma externe}
\label{chap:annexeC}
\hspace{0pt}

\lstset{language=XML}
\begin{lstlisting}

<?xml version="1.0"?>

<schema xmlns="http://www.w3.org/2001/XMLSchema">


<element name="pokemonData">

    <complexType>

    <sequence>

        <!-- TEAMS -->

        <element name="teams">

            <complexType>

            <sequence>

                <element name="team" maxOccurs="unbounded">

                    <complexType>

                    <sequence>

                        <element name="trainerName" type="string" />

                        <element name="victoryCounter" type="integer" />

                        <element name="defeatCounter" type="string" />

                        <element name="pokemon" minOccurs="1" maxOccurs="6">

                            <complexType>

                            <sequence>

                                <element name="name" type="string" />

                                <element name="nickname" type="string" />

                                <element name="height" type="string" />

                                <element name="weight" type="string" />

                                <element name="base_experience" type="string" />

                                <element name="type1" type="string" />

                                <element name="type2" type="string" minOccurs="0"/>

                                <element name="move" maxOccurs="4">

                                    <complexType>

                                    <sequence>

                                        <element name="spePhySta" type="string"/>

                                        <element name="power" type="integer" minOccurs="0" />

                                        <element name="accuracy" type="integer" minOccurs="0"/>

                                        <element name="pp" type="integer" />

                                        <element name="description" type="string" minOccurs="0"/>

                                    </sequence>

                                    <attribute name="id" type="string" use="required" />

                                    </complexType>

                                </element>

                            </sequence>

                            <attribute name="id" type="string" use="required" />

                            </complexType>

                        </element>

                    </sequence>

                    <attribute name="id" type="string" use="required" />

                    </complexType>

                </element>

            </sequence>

            </complexType>

        </element>


        <!-- MOVES -->

        <element name="moves">

            <complexType>

            <sequence>

                <element name="move" maxOccurs="unbounded">

                    <complexType>

                    <sequence>

                        <element name="spePhySta" type="string"/>

                        <element name="power" type="integer" minOccurs="0" />

                        <element name="accuracy" type="integer" minOccurs="0"/>

                        <element name="pp" type="integer" />

                        <element name="description" type="string" minOccurs="0"/>

                    </sequence>

                    <attribute name="id" type="string" use="required" />

                    </complexType>

                </element>

            </sequence>

            </complexType>

        </element>

        

        <!-- POKEDEX -->

        <element name="pokedex">

            <complexType>

            <sequence>

                <element name="pokemon" maxOccurs="unbounded">

                    <complexType>

                    <sequence>

                        <element name="name" type="string" />

                        <element name="height" type="string" />

                        <element name="weight" type="string" />

                        <element name="type1" type="string" />

                        <element name="type2" type="string" minOccurs="0"/>

                        <element name="base_experience" type="string" />

                    </sequence>

                    <attribute name="id" type="string" use="required" />

                    </complexType>

                </element>

            </sequence>

            </complexType>

        </element>


    </sequence>

    </complexType>

</element>

</schema>

\end{lstlisting}