La fédération de bases de données constitue un sujet central dans le contexte des systèmes d’information modernes. Alors que les bases de données se font de plus en plus variées, et leur utilisation essentielle chez toutes les entreprises, la problèmatique de fédération de bases de données différentes devient incontournable pour tout ingénieur informatique.

Ce projet est donc une première expérience dans l’intégration de bases de données dans un système fédéré. En effet, nous avons choisi de mettre en oeuvre la fédération de l’entrée d’une requête à la sortie de sa réponse. Cela nous donne ainsi une vue d’ensemble solide de ce qu’est une  fédération et de comment la mener à bien. Cependant, et en raison d’un temps imparti court, nous avons fait le choix de ne traiter que des requêtes simples. En effet, par souci de gain de temps, nous avons fait en sorte d’éviter tout conflit lié à, par exemple, l’hétérogénéité des langages de requêtes, des schémas, etc. en limitant la complexité des requêtes d’entrée.  Cela a peut-être impacté notre vision des difficultés liées à une fédération d’un point de vue pratique, mais nous avons tout de même pu étudier cet aspect à travers la mise en oeuvre, elle aussi pratique, de la fédération elle-même. 
Notre base de données fédérée remplit son rôle et permet à un utilisateur d’entrer une requête XQuery simple de type for...where...return, en lui retournant la réponse sous un format XML. Cependant, nous avons laissé des possibilités d’évolutions, telles que la prise en charge de requêtes d’entrée plus complexes, l’extension des sources de la fédération ou la mise en cache des requêtes les plus utilisées, pour de futurs développements.