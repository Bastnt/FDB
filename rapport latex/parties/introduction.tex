De nos jours, la plupart des entreprises peuvent être vues comme un ensemble de services, traitant des données hétérogènes en utilisant des technologies variées. De plus, ces mêmes données peuvent être stockées de manières différentes, via par exemple des bases de données relationnelles ou graphes, des fichiers XML, etc. L’un des rôles du système d’information est donc de structurer et gérer ces données. Cependant l’accès à ces données peut poser des difficultés. En effet, il est impensable de comparer ou de joindre des données issues de  sources différentes. Ce problème est pourtant inévitable puisque le contexte des systèmes d’informations modernes implique souvent des systèmes de gestion de bases de données différents au sein d’une même entreprise.

Pour pallier à cela, des architectures de bases de données permettent d'accéder à des données de nature complètement différente. Parmi ces architectures, les principales sont les bases de données fédérées et les “data warehouses” (entrepôts de données). Dans le cadre du projet de Bases de Données Avancées et Systèmes d’Information Modernes, nous étudierons le système de bases de données fédérée et nous mettrons en oeuvre une fédération de divers bases de données, telles qu’une base de données relationnelles et une base de données XML. Le but de ce travail est d'identifier les différentes étapes et modules qui entrent en jeu lorsqu'une requête est effectuée sur une base de données fédérée. Afin de comprendre les différents problèmes inhérents à la fédération, nous avons décidé de réaliser une telle base qui devra pouvoir traiter une requête, de l'émission de la requête par le client jusqu'à la production d'une réponse fédérée, en passant par le traitement dans chaque base. Cette requête se voudra simplifiée afin de pouvoir aborder tous les aspects possibles dans le court temps imparti.