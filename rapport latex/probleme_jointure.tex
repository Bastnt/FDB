Problème de jointure entre 2 sources distantes.
Les sources représentent les données de façon hétérogénes.


Nous représentons le problème sous forme de graphe biparti où les arêtes sont valuées.
On appelera S l'ensemble des sources, R l'ensemble des concepts, (i, j) les arêtes dans R x S et f(i, j) les valeurs des arêtes.
Les arêtes (i, j) du graphe signifient que le concept j est représenté dans la source i par le champ f(i, j).

Exemple:
Nous avons dans notre base
\begin{itemize}
	\item une table \textit{request\_teams}qui contient le champ \textit{id}
	\item un fichier XML \textit{teams.xml} qui contient l'attribut \textit{/teams/team/@id}
\end{itemize}
Le champ et l'attribut attribut correspond au concept \textit{team_id}. Nous avons donc:
\begin{itemize}
	\item f(\textit{request\_teams}, \textit{team_id}) = \textit{request\_teams.id}
	\item f(\textit{teams.xml}, \textit{team_id}) = \textit{request\_teams.id}
\end{itemize} 

Le problème de jointure entre deux éléments situés sur deux sources distinctes équivaut à trouver un chemin dans le graphe. Les valeurs des arêtes suivies donne les éléments/attributs/champs à utiliser.

Exemple:

Soient S = {A, B} et R= {a,b,c,d}
Soit A = {(A, a), (A, c), (B, a), (B, b), (A, d)}

